\pagenumbering{roman}
\setcounter{page}{1}
\pagestyle{plain}

%%%%%%%%%%%%%%%%
%%% CREDITS %%%
%%%%%%%%%%%%%%%%
\chapter*{Credits/Copyright}

The dataset used in this work is protected by the CC-BY-NC licence 
\cite{cc_by_nc_license}.

\begin{figure}[ht]
\centering
\includegraphics[scale=1]{images/CC BY-NC license.png}
\end{figure}

\vspace{1cm}

This document and its code are licensed under Attribution-NonCommercial-NoDerivs 3.0 Spain (CC BY-NC-ND 3.0 ES) 

\href{https://creativecommons.org/licenses/by-nc-nd/3.0/es/}{3.0 Spain of CreativeCommons}.


\begin{figure}[ht]
\centering
\includegraphics[scale=1]{images/license.png}
\end{figure}




%%%%%%%%%%%%%
%%% RECORD %%%
%%%%%%%%%%%%%
\chapter*{FINAL PROJECT RECORD}

\begin{table}[ht]
\centering{}
\renewcommand{\arraystretch}{2}
\begin{tabular}{r | l}
\hline
Title of the project: & Dermatological lesion classification\\
\hline
Author's name: & Jesús González Leal\\
\hline
Collaborating teacher's name: & Jordi de la Torre Gallart\\
\hline
PRA's name: & Laura Subirats Maté\\
\hline
Delivery date (mm/yyyy): & 01/2024\\
\hline
Degree or program: & Master’s degree in data science\\
\hline
Final Project area: & Medicine\\
\hline
Language of the project: & English\\
\hline
Code repository: & https://github.com/chus73/Dermatological-lesion-classification\\

\hline
Keywords & \textit Image classification, Transfer learning, SMOTE  \\
\hline
\end{tabular}
\end{table}

%%%%%%%%%%%%%%%%%%%
%%% DEDICATION %%%
%%%%%%%%%%%%%%%%%%%
\chapter*{Dedication/Quote}

Dedicado a mis pequeños, la fuente de mi alegría. Espero serviros de inspiración y que podáis lograr todos vuestros sueños. 

%%%%%%%%%%%%%%%%%%%
%%% Acknowledgements %%%
%%%%%%%%%%%%%%%%%%%
\chapter*{Acknowledgements}


    I would like to acknowledge the work of the International Sink Imaging Collaboration \cite{isic_web} in helping to reduce skin cancer mortality. 

I would also like to acknowledge the "Universitat Oberta de Catalunya" for giving me the opportunity to grow personally and professionally.

To conclude this chapter of acknowledgements, I would also like to include my English teacher, Jennifer, for her patience in helping me to make progress in learning English.

%%%%%%%%%%%%%%%%
%%% ABSTRACT %%%
%%%%%%%%%%%%%%%%
\chapter*{Abstract}
\addcontentsline{toc}{chapter}{Abstract}

\onehalfspacing

The skin is the largest organ in the body and the first barrier for defending our inner organs against aggression, besides helping the body regulate its temperature. Considering the importance of the skin for human beings, it is necessary to examine skin lesions. Because of that, dermatologists use dermatoscopy to illuminate and magnify the area to be examined to monitor any present injury.  

In recent decades, enormous advances in the AI medical field have experienced significant improvements due to the use of deep networks. They can diagnose with reliability as well as speed. 

The following project presents the use of deep learning algorithms in order to classify eight different diagnoses, such as melanoma, basal cell carcinoma, vascular lesion, and other lesions. In this study, we design and test two of today's most commonly used convolutional networks in the context of skin lesion classification, using the 2019 ISIC dataset. This dataset presented quality challenges, including resolution variations and very significant \textbf{inter-class imbalances}. We implemented two strategies to solve the dataset problem: an \textbf{under-sampling} and an \textbf{over-sampling} of the minority classes. We use \textbf{Synthetic minorities (SMOTE technique)} to enrich the minority classes, resulting in a 20 \% increase in the total number of images processed.

Furthermore, we employ model training transfer learning techniques using EfficientNet B0 and ResNet50 as a model base, demonstrating significant improvements in classification metrics. In particular, ResNet50 showed higher performance when trained on the SMOTE-enriched dataset, and this can be attributed to its 
 profound architecture benefiting from a larger dataset.

This research contributes to a new understanding of skin lesion classification through data preprocessing, transfer learning, and model selection, paving the way for future enhancements in dermatological image analysis and early diagnosis. Finally, it also contributes exploring the use of a statistic technique like SMOTE to solve an imbalanced dataset challenge.



\vspace{1.5cm}

\textbf{Keywords}: transfer learning, SMOTE, skin lesion classification, unbalance dataset.
