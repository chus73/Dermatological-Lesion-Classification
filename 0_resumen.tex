\pagenumbering{roman}
\setcounter{page}{1}
\pagestyle{plain}

%%%%%%%%%%%%%%%%
%%% CREDITS %%%
%%%%%%%%%%%%%%%%
\chapter*{Credits/Copyright}

A page with the specification of credits/copyright for the project (either application on one side and documentation on the other, or unified), as well as the use of third-party trademarks, products or services (including source code). If a person other than the author collaborated on the project, their identity and what they did must be explicitly stated.

Below is the most common case, but it can be modified for any other alternative:

\vspace{1cm}

\begin{figure}[ht]
\centering
\includegraphics[scale=1]{images/license.png}
\end{figure}

Attribution-NonCommercial-NoDerivs 3.0 Spain (CC BY-NC-ND 3.0 ES) 

\href{https://creativecommons.org/licenses/by-nc-nd/3.0/es/}{3.0 Spain of CreativeCommons}.

%%%%%%%%%%%%%
%%% RECORD %%%
%%%%%%%%%%%%%
\chapter*{FINAL PROJECT RECORD}

\begin{table}[ht]
\centering{}
\renewcommand{\arraystretch}{2}
\begin{tabular}{r | l}
\hline
Title of the project: & Dermatological lesion classification\\
\hline
Author's name: & Jesús González Leal\\
\hline
Collaborating teacher's name: & Jordi de la Tore\\
\hline
PRA's name: & Laura Subirats\\
\hline
Delivery date (mm/yyyy): & 02/2024\\
\hline
Degree or program: & Master’s degree in data science\\
\hline
Final Project area: & Medicine\\
\hline
Language of the project: & English\\
\hline
Code repository: & https://github.com/chus73/Dermatological-lesion-classification\\

\hline
Keywords & \textit Dermatological Diagnostic, Image classification  \\
\hline
\end{tabular}
\end{table}

%%%%%%%%%%%%%%%%%%%
%%% DEDICATION %%%
%%%%%%%%%%%%%%%%%%%
\chapter*{Dedication/Quote}

Dedicado a mis pequeños, la fuente de mi alegría. Espero serviros de inspiración y que podáis lograr todos vuestros sueños. 

%%%%%%%%%%%%%%%%%%%
%%% Acknowledgements %%%
%%%%%%%%%%%%%%%%%%%
\chapter*{Acknowledgements}

I would like to acknowledge the work of the International Sink Imaging Collaboration \cite{isic_web} in helping to reduce skin cancer mortality.

%%%%%%%%%%%%%%%%
%%% ABSTRACT %%%
%%%%%%%%%%%%%%%%
\chapter*{Abstract}
\addcontentsline{toc}{chapter}{Abstract}

\onehalfspacing

The skin is the largest organ in the body and the first barrier of defence against aggression from the outside world as well as helping the body to regulate its temperature. In order to examine skin lesions, dermatologists use dermatoscopy, which consists of illuminating and magnifying the area to be examined. 

The following project presents the use of deep learning algorithms in order to classify nine different diagnoses such as melanoma, basal cell carcinoma, vascular lesion and others of this kind.



\vspace{1.5cm}

\textbf{Keywords}: dermatoscopy, carcinoma, classification.
