\pagenumbering{roman}
\setcounter{page}{1}
\pagestyle{plain}

%%%%%%%%%%%%%%%%
%%% CREDITS %%%
%%%%%%%%%%%%%%%%
\chapter*{Credits/Copyright}

The dataset used in this work is protected by the CC-BY-NC licence 
\cite{cc_by_nc_license}.

\begin{figure}[ht]
\centering
\includegraphics[scale=1]{images/CC BY-NC license.png}
\end{figure}

\vspace{1cm}

This document, and its code, are licensed under Attribution-NonCommercial-NoDerivs 3.0 Spain (CC BY-NC-ND 3.0 ES) 

\href{https://creativecommons.org/licenses/by-nc-nd/3.0/es/}{3.0 Spain of CreativeCommons}.


\begin{figure}[ht]
\centering
\includegraphics[scale=1]{images/license.png}
\end{figure}




%%%%%%%%%%%%%
%%% RECORD %%%
%%%%%%%%%%%%%
\chapter*{FINAL PROJECT RECORD}

\begin{table}[ht]
\centering{}
\renewcommand{\arraystretch}{2}
\begin{tabular}{r | l}
\hline
Title of the project: & Dermatological lesion classification\\
\hline
Author's name: & Jesús González Leal\\
\hline
Collaborating teacher's name: & Jordi de la Torre Gallart\\
\hline
PRA's name: & Laura Subirats Maté\\
\hline
Delivery date (mm/yyyy): & 02/2024\\
\hline
Degree or program: & Master’s degree in data science\\
\hline
Final Project area: & Medicine\\
\hline
Language of the project: & English\\
\hline
Code repository: & https://github.com/chus73/Dermatological-lesion-classification\\

\hline
Keywords & \textit Dermatological Diagnostic, Image classification, CNN  \\
\hline
\end{tabular}
\end{table}

%%%%%%%%%%%%%%%%%%%
%%% DEDICATION %%%
%%%%%%%%%%%%%%%%%%%
\chapter*{Dedication/Quote}

Dedicado a mis pequeños, la fuente de mi alegría. Espero serviros de inspiración y que podáis lograr todos vuestros sueños. 

%%%%%%%%%%%%%%%%%%%
%%% Acknowledgements %%%
%%%%%%%%%%%%%%%%%%%
\chapter*{Acknowledgements}

I would like to acknowledge the work of the International Sink Imaging Collaboration \cite{isic_web} in helping to reduce skin cancer mortality. 

I would also like to acknowledge the "Universitat Oberta de Catalunya" for giving me the opportunity to grow personally and professionally.

%%%%%%%%%%%%%%%%
%%% ABSTRACT %%%
%%%%%%%%%%%%%%%%
\chapter*{Abstract}
\addcontentsline{toc}{chapter}{Abstract}

\onehalfspacing

The skin is the largest organ in the body, as well as the first barrier for defending our inner organs against any aggression, besides helping the body to regulate its temperature. Considering the importance of the skin for human beings, it is necessary to examine skin lesions and because of that dermatologists use dermatoscopy, which is a device for illuminating and magnifying the area to be examined, to monitor any present injury.  

During recent past decades, enormous advances in AI medical field have experienced important improvements due to the use of deep networks. They are able to diagnose with reliability as well as speed. 

The following project presents the use of deep learning algorithms in order to classify nine different diagnoses such as melanoma, basal cell carcinoma, vascular lesion and others of this kind.



\vspace{1.5cm}

\textbf{Keywords}: dermatoscopy, carcinoma, skin lesion classification, CNN.
